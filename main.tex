
\documentclass[12pt]{article}
\usepackage{mathptmx}
\usepackage{amsmath}
\usepackage{latexsym}
\usepackage{amsfonts}
\usepackage[normalem]{ulem}
\usepackage{soul}
\usepackage{array}
\usepackage{amssymb}
\usepackage{extarrows}
\usepackage{graphicx}

\usepackage[utf8]{inputenc}

% bib
\usepackage[british]{babel}
\usepackage[style=british]{csquotes}
\usepackage[
    backend=biber,
    firstinits=false,
    style=ieee,
    sorting=none,
    doi=false
]{biblatex}
\usepackage{biblatex}
\addbibresource{refs.bib}

\DeclareFieldFormat{labelnumberwidth}{#1\adddot}
\setlength{\biblabelsep}{5pt}

\DeclareFieldFormat{journaltitle}{#1}
\DeclareFieldFormat[article]{volume}{Vol\addnbspace #1}
\DeclareFieldFormat[article]{number}{No \addnbspace #1}
\DeclareFieldFormat[article]{pages}{pp \addnbspace #1}

\usepackage{subfig}
\usepackage{wrapfig}
\usepackage{wasysym}
\usepackage{enumitem}
\usepackage{adjustbox}
\usepackage{ragged2e}
\usepackage[svgnames,table]{xcolor}
\usepackage{tikz}
\usepackage{longtable}
\usepackage{changepage}
\usepackage{setspace}
\usepackage{hhline}
\usepackage{multicol}
\usepackage{tabto}
\usepackage{float}
\usepackage{multirow}
\usepackage{makecell}
\usepackage{fancyhdr}
\usepackage[toc,page]{appendix}
\usepackage[hidelinks]{hyperref}
\usetikzlibrary{shapes.symbols,shapes.geometric,shadows,arrows.meta}
\tikzset{>={Latex[width=1.5mm,length=2mm]}}
\usepackage{flowchart}\usepackage[paperheight=11.69in,paperwidth=8.27in,left=0.98in,right=0.98in,top=1.48in,bottom=1.48in,headheight=1in]{geometry}
\usepackage[T1]{fontenc}

\urlstyle{same}

\renewcommand{\_}{\kern-1.5pt\textunderscore\kern-1.5pt}


\pagenumbering{gobble}
\setlength{\topsep}{0pt}\setlength{\parindent}{0pt}
\renewcommand{\arraystretch}{1.3}

\usepackage{secdot}

\usepackage{titlesec}
\titleformat*{\subsection}{\large\bfseries\itshape}
\titleformat*{\subsubsection}{\normalfont\itshape}


%%%%%%%%%%%%%%%%%%%% Document code starts here %%%%%%%%%%%%%%%%%%%%

\begin{document}
\section*{\centering NDT.net Guidelines for the Preparation of Manuscripts for Conference Proceedings}

\vspace{\baselineskip}
\begin{Center}
Andrew G. FAMILYNAME1 \textsuperscript{1}, Peter FAMILYNAME2 \textsuperscript{2}, David FAMILYNAME3 \textsuperscript{1}
\end{Center}

\vspace{\baselineskip}
\begin{Center}
\textsuperscript{1}{Department of University, University/Company; City, Country}
\end{Center}
\begin{Center}
{ e-mail: author1@company.com, e-mail: author3@company.com}
\end{Center}
\begin{Center}
\textsuperscript{2}{Department, Company/University; City, Country, e-mail: author2@company.com}
\end{Center}

\vspace{\baselineskip}
\begin{FlushLeft}
\textbf{Abstract}
\end{FlushLeft}
{ These instructions have been prepared to assist authors in the preparation of papers for reproduction in conference proceedings to be provided to delegates. The instructions should be followed in all matters of format including section headings, capitalisation, punctuation, table and figure headings and their placement within the text. The conference proceedings will be available online after the conference as a part of open-access conference proceedings. These guidelines are to ensure maximum uniformity of style and reproduction without further modifications - please try to follow them as closely as possible. The material that you supply will be used exactly as it is presented. Your paper should not exceed 10 pages..\par}

\vspace{\baselineskip}
{ \textbf{Keywords:} Laser ultrasound, time of flight (TOF), welding, aerospace, carbon fiber composite}

\section{Introduction}
In preparing a manuscript, authors are solely responsible for the quality and appearance of the final product. Please follow these guidelines \cite{Udpa2020} carefully and accurately. 

\bigskip
The conference may request camera ready MS-Word or PDF files. In any case only PDF files will be published in the proceedings.

.

If any questions or special problems arise, feel free to contact the conference.

\section{Specific instructions}

\subsection{Text}

\subsection{Format}

\subsubsection{Title}

\printbibliography

\end{document}